\part*{Geometri}


\setcounter{section}{0}
\section{Mätning och rumsuppfattning}

\begin{outline}
  \1 Steg 1
    \2 Grundläggande förståelse för rum och tid.
    \2 Förståelse för principen att mäta något genom att jämföra med en fast enhet.
    \2 Vanliga lägesord, till exempel över, bakom och inuti.
    \2 Veckodagar, månader och årstider; namn, ordning och kännetecken. Namn och kännetecken för olika tider på dygnet.
  \1 Steg 2
    \2 Grundläggande hantering av storheterna längd, massa, volym och tid.
    \2 Mätning av dessa storheter.
    \2 Vanliga enheter av dessa storheter.
    \2 Grundläggande växling mellan olika enheter för storheterna (så som 60 minuter = 1 timme och 1 l = 10 dl).
    \2 Uppskattning och rimlighetsbedömning av dessa storheter.
    \2 Läsa av tiden på digitala och analoga klockor.
    \2 Höger och vänster. (Med- och moturs?)
  \1 Steg 3
    \2 Area: mätning, uppskattning och rimlighetsbedömning av, samt enheter för. Grundläggande växling mellan olika enheter (så som 100 cm^2 = 1 dm^2).
    \2 Metoder för att uppskatta och beräkna omkrets av plangeometriska objekt, inklusive användning av grundläggande formler.
    \2 Metoder för att uppskatta och beräkna area för tvådimensionella geometriska objekt, inklusive användning av grundläggande formler.
    \2 Metoder för att uppskatta och beräkna volym för tredimensionella geometriska objekt, inklusive användning av grundläggande formler. Volym uttryckt i m^3, dm^3 och cm^3.
    \2 Vinkel: mätning, uppskattning och rimlighetsbedömning av, samt enheten grader. Begreppen rät, spetsig och trubbig vinkel.
  \1 Steg 4
    \2 Beräkningar med enhetsbyten samt situationer där olika enheter blandas.
    \2 Skala i en, två och tre dimensioner.
\end{outline}
